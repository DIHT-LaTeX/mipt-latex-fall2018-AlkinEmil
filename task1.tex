\documentclass[12pt]{article}
\usepackage[russian]{babel}
\usepackage[utf8]{inputenc}	

\title{Домашняя работа №1}
\author{Эмиль Алкин}
\date{}

\begin{document}
	\maketitle
	{\itshape \hfill Audi multa, \par
	\hfill loquere pauca} \par
	\vspace{20 pt}
	Это мой первый документ в системе компьютерной вёрстки \LaTeX .\\
	\begin{center}
	{\sffamily \LARGE <<Ура!!!>>}
	\end{center}
	
	А теперь формулы. {\scshape Формула}~--- краткое и точное словесное выражение, определение или же ряд математических величин, выраженный условными знаками.\\[15pt]
	\hspace*{28pt} {\bfseries \Large Термодинамика} \par
	Уравнение Менделеева--Клайперона~--- уравнение состояния идеального газа, имеющее вид $pV=\nu RT$, где $p$~--- давление, $V$~--- объём, занимаемый газом, $T$~--- температура газа, $\nu$~--- количество вещества газа, а $R$~--- универсальная газовая постоянная.\\[15pt]
	\hspace*{28pt} {\bfseries \Large Геометрия \hfill Планиметрия}\par
	Для {\itshape плоского} треугольника со сторонами $a, b, c$ и углом $\alpha$, лежащим против стороны $a$, справедливо соотношение 
	$$a^2 = b^2 + c^2 - 2bc\cos{\alpha},$$ 
	из которого можно выразить косинус угла треугольника: 
	$$\cos{\alpha}=\frac{b^2+c^2-a^2}{2bc}.$$
	
	Пусть $p$~--- полупериметр треугольника, тогда путём несложных преобразований можно получить, что 
	$$\tg{\frac{\alpha}{2}}=\sqrt{\frac{(p - b)(p - c)}{p(p - a)}},$$
	\vspace{1cm}
	На сегодня, пожалуй, хватит\dots Удачи!
\end{document}
