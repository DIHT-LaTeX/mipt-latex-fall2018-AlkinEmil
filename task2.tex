\documentclass[11pt]{article}
\usepackage [
	a5paper,
	left = 1 cm, 
	right = 1 cm, 
	top = 1 cm, 
	bottom = 1 cm
]{geometry}
%\usepackage{showframe}
\usepackage[russian]{babel}
\usepackage[utf8]{inputenc}	
\usepackage{amsmath}
\usepackage{amssymb}
\usepackage{amsthm}
\usepackage{dsfont}
\usepackage{gensymb}
\DeclareMathOperator{\mat}{Mat}
\begin{document}
\newtheorem{definition}{Определение}
\newtheorem{thm}{Теорема}[section]
\newcounter{sum}
\stepcounter{sum}
\newenvironment{problem}[1]{\vspace{3pt} {\sffamily\bfseries\thesum. #1} \par}%
{\vspace{12pt} \\ \stepcounter{sum}}
	\section{Центральная предельная теорема}
	\begin{thm}[Линдеберга]
	Пусть $\{\xi_k\}_{k\geq1}$~--- независивые случайные величины, $\mathsf{E}\xi_k^2<+\infty$ $\forall k$, обозначим $m_k = \mathsf{E}\xi_k$, $\delta_k^2 = \mathsf{D}\xi_k > 0: S_n = \sum\limits_{i = 0}^n \xi_i $; $\mathsf{D}_n^2 = \sum\limits_{k = 1}^n \delta_k^2$ и $F_k(x)$~--- функция распределения $\xi_k$. Пусть выполнено условие Линдеберга, то есть
  	$$\forall\mathcal{E}>0 \frac{1}{\mathsf{D}_n^2} \sum  \limits_{k = 1}^n \int \limits_{\{x:|x-m_k|>\mathcal{E} \mathsf{D}_n\}} (x-m_k)^2 \, dx \xrightarrow[n\rightarrow\infty]{}0.$$ 
  	Тогда $\frac{S_n-ES_n}{\sqrt{DS_n}}\xrightarrow{d} \mathcal{N}(0,1), n\rightarrow\infty.$
	\end{thm}
	\section{Гауссовские случайные векторы}
	\begin{definition}
	Случайный вектор~$\vec{\xi}$~--- гауссовский, если его характеристическая функция
	$\varphi_{\vec{\xi}\,} (\vec{t}\,) = \exp\left( i(\vec{m}, \vec{t}\,) - \frac{1}{2}(\Sigma \vec{t},\vec{t}\,)\right), \vec{m}\in\mathbb{R}^n, \Sigma$~--- симметрическая неотрицательно определенная матрица.
	\end{definition}

	\begin{definition}
	Случайный вектор~$\vec{\xi}$~--- гауссовский, если он представляется в следующем виде: 
	$\vec{\xi} = A\vec{\eta} + \vec{b}$, где $\vec{b} \in \mathbb{R}^n$, $A\in \mat(n\times m)$ и $\vec{\eta} = (\eta_1, \dots , \eta_m)$~--- независимые и $\mathcal{N}(0,1)$.
	\end{definition}
	
	\begin{definition}
	Случайный вектор~$\vec{\xi}$~--- гауссовский, если $\forall\lambda\in\mathbb{R}^n$ случайная величина~$(\vec{\lambda},\vec{\xi})$ имеет нормальное распределение.
	\end{definition}
	
	\begin{thm}[об эквивалентности определений гауссовских векторов]
	Предыдущие три определения эквивалентны.
	\end{thm}
    \section{Задачи по астрономии}
    \begin{problem} {Dark Matters}
    В некотором скоплении галактик содержится $70$ спиральных и $30$ эллиптических галактик. Известно, что абсолютная звездная величина эллиптических галактик равна  $-20$, соотношение масса--светимость составляет $15\mathfrak{M}_\odot / L_\odot$. У спиральных галактик  в данном скоплении максимальная скорость вращения составляет $210$ км/с, соотношение масса--светимость~--- $5\mathfrak{M}_\odot / L_\odot$.\par
    Оцените долю темной материи внутри скопления, если масса межгалактического газа на порядок превышает массу галактик, а типичные скорости галактик в скоплении составляют $1000$ км/с. Размер скопления составляет $7$ Мпк. Абсолютная звёздная величина Млечного Пути~--- $-20.9$.
    \end{problem}
    \begin{problem} {Бейрут}
    В какой момент по истинному солнечному времени $1$ сентября Регул $(\alpha_1 = 10^\text h\,9^\text m, \delta_1 = 11\degree\,53')$ и Шератан $(\alpha_2 = 11^\text h\,15^\text m, \delta_2 = 15\degree\,20')$ находятся на одном альмукантарате в Бейруте $(\varphi = 33\degree\,53')$.
    \end{problem}
    \begin{problem} {H II}
    Предположим, что за пределами солнечного круга кривая вращения галактики плоская, параметр плато $v = 240$ км/с. Пусть известно, что диск нейтрального водорода на галактической долготе $l = 140\degree$. Оцените минимально возможное значение лучевой скорости этого облака.
    \end{problem}
    \begin{problem} {Обратный комптон--эффект}
    Обратным эффектом Комптона (ОЭК) называют явление рассеяния фотона на ультрарелятивистском свободном электроне, при котором происходит  перенос энергии от электрона к фотону. Рассмотрите ОЭК для фотонов реликтового излучения. При какой энергии электронов в направленном пучке рессеянное излучение можно будет зарегистрировать на фотоприёмнике?
    \end{problem}
    \section{Отзыв}
    \renewcommand{\labelitemi}{$\diamond$}
    \begin{itemize}
    	\item Курс интересный и полезный
    	\item Хотелось бы побольше примеров использования новых команд
    \end{itemize}
\end{document}